\documentclass[uplatex,dvipdfmx]{jsarticle}

\usepackage[uplatex,deluxe]{otf} % UTF
\usepackage[noalphabet]{pxchfon} % must be after otf package
\usepackage{stix2} % 欧文&数式フォント
\usepackage[fleqn,tbtags]{mathtools} % 数式関連 (w/ amsmath)
\usepackage{hyperref}

\title{Web掲示板システム仕様書}
\author{小川 泰生}
\date{\today}

\begin{document}

\maketitle

\tableofcontents

\section{はじめに}
本レポートは,Web掲示板システムについて,利用者向けガイド,管理者向けガイド,および開発者向けガイドの3つの視点から構成されています.以下に,それぞれのセクションの内容について説明します.

\subsection{レポートの構成概要}
\begin{enumerate}
    \item 利用者向けガイド:
    \begin{itemize}
        \item アプリケーションの目的や機能一覧を説明し,ユーザーがどのようにアプリケーションを操作するかを示す,タスクの追加・削除・完了状態の変更など,主要な機能についての操作方法を具体的に説明する.
    \end{itemize}
    \item 管理者向けガイド:
    \begin{itemize}
        \item サーバーセットアップやシステム要件について説明し,管理者がアプリケーションを正常に運用するために必要な手順を記載する.
    \end{itemize}
    \item 開発者向けガイド:
    \begin{itemize}
        \item プロジェクトのディレクトリ構造,ファイル構成,使用した技術について説明する.
        \item フロントエンドとバックエンド間のAPI通信の仕様(エンドポイント,データ形式,リクエストとレスポンスの例など)を記載する.
        \item 将来的な拡張案や改善点についても提案する.
    \end{itemize}
\end{enumerate}

\section{利用者向けガイド}
\subsection{アプリケーションの目的}
Web掲示板システムは,ユーザーがメッセージを投稿,表示,ソート,削除できるよう設計されています.ユーザーが簡単に操作できるインターフェイスを提供し,コミュニケーションを円滑にします.

\subsection{画面のレイアウトとボタンの仕様}
掲示板システムの画面は以下の要素で構成されています.
\begin{itemize}
    \item \textbf{ヘッダー}: システムのタイトルを表示します.
    \item \textbf{入力フィールド}: 名前とメッセージを入力するためのテキストフィールドです.
    \item \textbf{送信ボタン}: 入力されたメッセージを掲示板に追加します.
    \item \textbf{リセットボタン}: すべての投稿を削除します.
    \item \textbf{ソートボタン}: メッセージを名前順でソートします.
    \item \textbf{投稿表示ボタン}: すべての投稿を表示します.
    \item \textbf{文字数カウント}: メッセージの文字数をリアルタイムで表示します.
\end{itemize}

\section{管理者向けガイド}
\subsection{サーバーのセットアップ}
このアプリケーションは,ターミナルを使用して\texttt{app8.js}を起動することでサーバーが動作します.以下の手順に従って、サーバーをセットアップします.
\begin{enumerate}
    \item \texttt{app8.js}を使用したサーバーの起動
    \begin{verbatim}
    node app8.js
    \end{verbatim}
    このコマンドを実行すると,サーバーが起動し,アプリケーションが動作します.

    \item サーバーの動作確認
    ブラウザで以下のURLにアクセスして、掲示板が正常に動作することを確認します.
    \begin{verbatim}
    http://localhost:8080/public/bbs.html
    \end{verbatim}
\end{enumerate}

\section{開発者向けガイド}
\subsection{内部的な作りと変数の意味}
このアプリケーションの主な内部構造と変数の意味を以下に示します.
\begin{itemize}
    \item \textbf{mainContainer}: メッセージリストを格納する要素.
    \item \textbf{nameInput}: 名前を入力するテキストフィールド.
    \item \textbf{messageInput}: メッセージを入力するテキストフィールド.
    \item \textbf{charCount}: メッセージの文字数をカウントする要素.
    \item \textbf{addMessage}: メッセージをリストに追加する関数.
\end{itemize}

\subsection{通信内容}
フロントエンドとバックエンドの通信は,以下のエンドポイントを使用して行われます.
\begin{itemize}
    \item \textbf{GET /messages}: すべてのメッセージを取得.
    \item \textbf{POST /messages}: 新しいメッセージを追加.
    \item \textbf{DELETE /messages/:id}: 指定されたIDのメッセージを削除.
\end{itemize}

\subsubsection{リクエスト例}
\begin{verbatim}
POST /messages
{
  "name": "ユーザー名",
  "message": "メッセージ内容"
}
\end{verbatim}

\subsubsection{レスポンス例}
\begin{verbatim}
{
  "id": 1,
  "name": "ユーザー名",
  "message": "メッセージ内容",
  "timestamp": "2025-01-07T12:00:00Z"
}
\end{verbatim}

\subsection{データ形式の詳細}
メッセージデータはJSON形式でやり取りされます.以下に,主要なデータフィールドとその説明を示します.
\begin{itemize}
    \item \textbf{id}: メッセージの一意の識別子(整数)
    \item \textbf{name}: メッセージを投稿したユーザーの名前(文字列)
    \item \textbf{message}: 投稿されたメッセージ内容(文字列)
    \item \textbf{timestamp}: メッセージが投稿された日時(ISO 8601形式の文字列)
\end{itemize}

\subsection{通信フロー}
メッセージの送信と取得の基本的なフローは以下の通りです.
\begin{enumerate}
    \item ユーザーがメッセージを入力し,送信ボタンをクリックすると,フロントエンドからバックエンドにPOSTリクエストが送信されます.
    \item バックエンドはリクエストを受け取り,新しいメッセージをデータベースに保存し,保存されたメッセージのデータをレスポンスとして返します.
    \item フロントエンドはレスポンスを受け取り,新しいメッセージを画面に表示します.
\end{enumerate}

\section{GitHubリポジトリ}
このアプリケーションのソースコードはGitHubでホストされています.以下のURLからアクセスできます:
\url{https://github.com/Ogitai/SPA.git}

\end{document}


